\documentclass[main.tex]{subfiles}
\begin{document}
\begin{abstract}
%context, gap and contribution
%\todoL{Finish abstract}
\noindent The purpose of the project on the 7\textsuperscript{th} semester of Software Engineering was to identify and enable better privacy control and awareness in mobile sensing applications. This aim was realized by creating an Android application prototype to help the user enforce privacy. In order to prevent tracking the application Yet Another Network Filter (YANF) was created. YANF works by creating a network filter enabling the removal of cookies and advertisements. YANF differs from already existing solutions by creating global control of the network for all browsers and applications on the device. This is done without having to root the device. Quality Attribute Scenarios was created to ensure a certain level of quality where response measures has been tested and measured. The conducted tests shows that the architecture for YANF is suitable of meeting the current requirements in regards to privacy and energy efficiency. The project resulted in a prototype for further development.


%Implementation of tactics and adaption of testing these also plays a key role in the project. 

%This paper aims to examine and demonstrate the process of developing an Android application that enables privacy through removal of cookies. The technology specifically chosen for this is the NetBare library.  




% Context: Privacy, tracking, Android, CookieBot?,
% Gap: no globally cookie/ad removing solutions. 
% Contribution: Using NetBare?, QAS, Tactics, blacklist

\end{abstract}
\end{document}